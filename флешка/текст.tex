\documentclass[12pt, a4paper]{article}

\usepackage[T1, T2A]{fontenc}
\usepackage[utf8]{inputenc}
\usepackage[english, russian]{babel}
\usepackage{amsmath, amssymb}
\usepackage[left=3cm, right=1.5cm, top=2cm, bottom=2cm]{geometry}
\usepackage[displaymath]{lineno}
\usepackage[space]{cite}
\usepackage{xcolor, graphicx}

\renewcommand{\vec}{\mathbf}
\linenumbers
\linespread{2}

\def \eps {\varepsilon}
\def \w {\omega}
\def \ph {\varphi}
\def \kp { \varkappa}
\newcommand{\dt}[1]{\frac{\partial {#1}}{\partial t}}
\newcommand{\dx}[1]{\frac{\partial {#1}}{\partial x}}
\newcommand{\dn}[1]{\left.\frac{\partial #1}{\partial \vec{n}}\right|_{ S}}
\newcommand{\dr}[1]{\left.\frac{\partial #1}{\partial r}\right|_{ r = a}}
\newcommand{\sumexp}[1]{\frac{{#1} e^{i \w t} + {#1}^* e^{-i \w t}}{2}}
\newcommand{\sumexptwo}[1]{\frac{{#1} e^{2 i \w t} + {#1}^* e^{-2 i \w t}}{2}}
\newcommand{\dtau}[1]{\frac{\partial {#1}}{\partial \tau}}
\begin{document}


\section*{В\,В\,Е\,Д\,Е\,Н\,И\,Е}

Металлические наноструктуры привлекают к себе большое внимание благодаря своим уникальным характеристикам, связанным с возможностью возбуждения в них плазмонных резонансов на частоте падающего на наночастицу электромагнитного излучения.
Основной интерес к таким плазмонным наноструктурам обусловлен их уникальной способностью локализовать электромагнитные поля на нанометровых масштабах, существенно меньших дифракционного предела, что позволяет контролировать свойства света в размерах, намного меньших его длины волны.[A]

Благодаря плазмонным резонансам в наноструктурах происходит существенное увеличение локальной плотности энергии поля, что приводит к возможности проявления в них различного рода нелинейных эффектов, включающих многофотонную люминесценцию [], четырехволновое смешивание [], и генерацию гармоник оптического излучения [вт гарм тр гарм].

В частности, явление генерации второй гармоники в наноструктурах, возможность возникновения которого в ограниченных металлических объектах была впервые обнаружена экспериментально и объяснена теоретически в работах [5, 6], является в настоящее время основой для широкого круга практических применений, включающего диагностику наноструктур [см эксп обзор] и оптических сред [7].

Важным фактором, благодаря которому наноструктуры и основанные на них метаматериалы могут служить эффективным инструментом для генерации второй гармоники, является возможность резонансного усиления поля не только основной гармоники оптического излучения, но и его второй гармоники при совпадении удвоенной частоты с собственной частотой другой плазмонной модой наноструктуры. 

К настоящему моменту явление двойного плазмонного резонанса исследовалось фактически только для наноструктур обеспечивающих одновременное возбуждение двух различных поверхностных плазмонов наночастицы на основной и удвоенной гармониках падающего излучения.

Однако в общем случае в наноструктуре, помимо поверхностных плазмонов могут существовать и объемные плазмоны [] ¬ моды коллективных электронных колебаний, представляющие собой стоячие плазменные (Ленгмюровские) волны и возникающие из-за пространственной дисперсии (нелокальности поляризуемости плазмы). Объемные плазмоны, как известно, могут сильно проявлять себя в случае, когда источник возбуждения коллективных электронных колебаний находится внутри наночастицы и характеризуется неоднородным распределением поля, что, например имеет место в задачах спектроскопии характеристических потерь энергии электронами (англ. Electron Energy Loss Spectroscopy) при рассеянии пучков заряженных частиц наноструктурами. 

Подобная ситуация может возникнуть и в задачах генерации второй гармоники, когда обусловленные нелинейностью токи второй гармоники, возбуждаемые при резонансе поверхностного плазмона на основной частоте колебаний, могут возбуждать объемные плазмонные колебания в наночастице. Данный эффект может иметь место, например, в случае наноструктуры простейшей формы, металлической сферической наночастицы, однако к настоящему моменту двойные плазмонные резонансы типа поверхностный плазмон ¬ объемный плазмон фактически не были исследованы и являются предметом исследования данной работы.

В данной работе на основании гидродинамической модели [] исследуются нелинейные эффекты, обусловленные возникновением резонансов объемных плазмонов на удвоенной частоте в условиях, когда частота основной гармоники наночастицы также испытывает резонанс и совпадает с частотой дипольного поверхностного плазмона наночастицы (хорошо известный резонанс Ми). Работа организована следующим образом: вначале на основе уравнений гидродинамики с использованием метода последовательных приближений сформулированы краевые задачи, описывающие в квазистатическом приближении пространственное распределение поля и плотности заряда на основной и удвоенной гармониках внешнего поля в малой металлической наночастице произвольной формы. Далее описано решение этих задач применительно к случаю сферической наночастицы, и исследованы условия отвечающие условию возбуждения в наночастицах двойных резонансов типа поверхностный плазмон – объемный плазмон. После приводятся результаты расчетов, иллюстрирующие влияние исследуемых резонансов на частотные зависимости сечения поглощения сферических наночастиц и сформулированы основные результаты работы.


\section{ПОСТАНОВКА ЗАДАЧИ}

Рассмотрим металлическую наночастицу произвольной формы, находящуюся в заданном внешнем поле падающей электромагнитной волны, и находящуюся с среде с диэлектрической проницаемостью $\eps_d$. Как известно, достаточно подробное описание нелинейной динамики носителей в квазиклассическом приближении может быть получено с помощью набора уравнений гидродинамики (уравнение непрерывности и уравнение Эйлера), описывающих электронную плазму как сжимаемую заряженную жидкость [ОБЗ ТЕОР ГД  12–15]. 

При дальнейшем построении физической модели исследуемых двойных резонансов будем считать выполненными ряд приближений, а именно будем предполагать, что (I) размеры наночастицы малы по сравнению с длиной падающей волны и допустимо квазистатическое приближение для описания поля внутри и вблизи поверхности наночастицы (II) вклад в магнитную составляющую силы Лоренца, действующую на электроны в металле пренебрежимо мал, (III) электроны находятся внутри бесконечно глубокой потенциальной ямы, то есть будем пренебрегать эффектом размывания профиля электронной плотности близ границы металла (так называемый spill-out effect) [], возникающим при учете давления электронов и (IV) положительный заряд ионного остова с равномерной плотностью распределен по объему наночастицы (предполагается, что в отсутствие внешнего поля электроны, как и ионы, распределены равномерно по объему частицы с плотностью $N_0$, а диэлектрическая проницаемость ионного остова материала частицы равна $\eps_\infty$): 
\begin{equation} 
\frac{v_F}{\omega_p} \ll L \ll \frac{2\pi c}{2\omega\sqrt{\eps_{d, \infty}}}  \quad v \ll c, 
\end{equation} 
где $v_F = \hbar (3 \pi^2 N_0)^\frac{1}{3}/m $ — скорость Ферми, $c$ — скорость света, $e$ и $m$ — заряд и масса электрона, $\hbar$ — постоянная Планка, $L$ — характерный размер частицы, $\w$ — частота падающего поля, $ w_p^2 = 4 \pi e^2 N_0 / m$  — плазменная частота.

Вместе с условиями применимости гидродинамического подхода указанные выше условия несколько сужают область применимости рассматриваемой модели, однако поскольку ранее двойные плазмонные резонансы обсуждаемого здесь типа фактически не исследовались, такое упрощение модели представляется оправданным первым шагом на пути построения более точной модели. Таким образом, с учетом указанных предположений, нелинейная динамика коллективных электронных колебаний в наночастице подчиняется системе уравнений:
\begin{equation} 
	\label{непрерывности}
	\dt{N} + \operatorname{div}(N \vec{v}) = 0,
\end{equation}
\begin{equation} 
	\label{гидродинамики}
	\dt{\vec{v}} + \nu \vec{v} +(\vec{v} \nabla)\vec{v} = \frac{e}{m}\vec{E} - \frac{1}{mN} \nabla p, 
\end{equation}
где $\vec{v}$ – скорость электронов, $N$ – возмущённая концентрация электронов, $\nu$ – эффективная частота соударений электронов, $\vec{f} = N \vec{v}$ имеет смысл потока электронов, $p$ – давление электронов. Конкретный вид выражения для последней из перечисленных величин, фактически отвечающей за нелокальность поляризационного отклика плазмы, являлся предметом множества дискуссий и в настоящее время существует широкий спектр моделей, описывающих эту величину применительно к различным условиям. В рамках рассматриваемой здесь простой модели мы используем следующе феноменологическое уравнение состояния, отвечающее исследуемому здесь случаю быстрого адиабатического процесса и позволяющее получить из описанных выше уравнений () известный закон дисперсии как для поверхностных, так и для объемных плазмонов:  $p = p_0 (N/N_0)^\gamma$, где $p_0 = m v_f^2 N_0/5$, $\gamma = 3$.

Следуя обычной процедуре метода возмущений, применяемого в случае слабой нелинейности, представим в уравнениях неизвестные плотность электронов, скорость и напряженность поля в виде суммы гармонических слагаемых, изменяющихся на частоте, кратной частоте внешнего поля. Далее сопоставляя в получившихся уравнениях величины одинакового порядка малости, получаем следующие уравнения, определяющие комплексные амплитуды плотности заряда и потенциала поля для основной ($\w_1=\w$, $n=1$) и удвоенной ($\w_2=2\w$, $n=2$) гармоник.
\begin{equation} 
 \Delta \rho_n + k_p^2(\w_n)\rho_n = -\frac{1}{4 \pi r_0^2} \Delta \ph^{ex} + \frac{\w_n(\w_n - i \nu)}{\w_p^2r_0^2} \rho^{ex}
\end{equation}
\begin{equation} 
 \Delta \ph_n = - \frac{4 \pi}{\eps_\infty} \rho_n, \quad n = 1,2. 
\end{equation}
ОБОЗНАЧЕНИЕ R0 ВВЕСТИ
Введенные в уравнениях обозначения $\ph^{ex}$ и $\rho^{ex}$ играют фактически роль расположенных внутри плазмы сторонних источников колебаний. Для первой гармоники они, очевидно, тождественно равны нулю и введены только для более краткой и единой записи результирующих уравнений. Для колебаний второй гармоники выражения для источников определяется выражениями 
\begin{equation}
 - 2i\w \rho^{ex} = \frac{1}{2}\operatorname{div} \rho_1 \vec{v_1},
\end{equation}
\begin{equation}
\ph^{ex} = \frac{m}{4e}(\frac{v_0^2}{N_0^2}N_1^2 + \vec{v}_1^2),
\end{equation}
и фактически имеют смысл сторонней осциллирующей плотности заряда, (возникающей из-за нелинейного слагаемого в уравнении непрерывности ()) и потенциала стороннего поля, определяющего дополнительную силу, действующую на заряды плазмы на удвоенной частоте (возникающего из-за нелинейности уравнения состояния () и из-за конвективного члена в уравнении () ). 

Система уравнений () должна быть дополнена граничными условиями на поверхности наночастицы. Первые из используемых нами граничных условий, вытекают непосредственно из уравнений Максвелла
\begin{equation} 
	\left. \ph_n \right|_{ S} = \left. \ph_n^{out} \right|_{ S} 
\end{equation}
\begin{equation} 
	\eps_\infty \dn{\ph_n} = \eps_d \dn{\ph_n^{out}}, \quad n = 1,2,  
\end{equation}
и связывают потенциалы электрического поля внутри наночастицы с соответствующими потенциалами $\ph_{1,2}^{out}$ в окружающем ее однородном диэлектрике, удовлетворяющими уравнению ЧЧЧ.
Последнее, необходимое для однозначного решения сформулированных уравнений, граничное условие определяется характером движения электронов близ границы наночастицы. В случае принимаемого здесь условия зеркального отражения электронов от поверхности металла соответствующее граничное условие принимает вид, 
\begin{equation} 
\vec{v}_n = -\frac{e}{i(\w_n - i\nu)m} \nabla \psi_n
\end{equation}
\begin{equation} 
\dn{\psi_n}	= 0, \quad \psi_n = \ph_n + 4 \pi r_0^2 \rho_n + \ph^{ex}, \quad n = 1,2,  
\end{equation}
где $\psi_{1,2}$ фактически имеют смысл потенциала скорости электронов на основной и удвоенной гармониках колебаний.

Сформулированная система уравнений (), как и в других подобных работах, посвященных исследованию генерации второй гармоники в условиях двойных резонансов, позволяет рассчитать структуру колебаний []. Новым основным новым элементом здесь является здесь учет нелокальности поляризации плазмы не только для основной, но и для удвоенной гармоники, что позволяет описать возникновение резонансов объемных плазмонов на этой частоте. Как известно, поле объемных плазмонов сильно локализовано внутри наночастицы и соответствующие им резонансы обычно слабо проявляется в спектрах рассеянного излучения, однако как будет показано далее, возбуждение объемных плазмонов на удвоенной частоте может приводить к заметному изменению поглощаемой наночастицей мощности. Расчет спектров поглощения в рамках рассматриваемой модели может быть выполнен следующим образом. Потери энергии обусловлены наличием в уравнении (1.2) диссипативной силы, с плотностью $\mu = m \nu \vec{f}$. Средняя за период плотность мощности этой силы очевидным образом может быть выражена через комплексные амплитуды плотностей потока и скоростей первой и второй гармоник. Интегрируя по объему наночастицы V с учетом соотношений () и граничного условия (), приходим к следующему выражению для средней за период мощности потерь во всем объеме наночастицы: 
\begin{equation} 
Q = \frac{\nu}{2} Re \iiint (\frac{\w}{i(\w - i \nu)}\rho_1 \psi_1^* + \frac{2\w}{i(2\w - i \nu)}\rho_2 \psi_2^*)dV.
\end{equation}

\section{СФЕРИЧЕСКАЯ НАНОЧАСТИЦА}
Применительно к сферической наночастице радиуса а, помещенной в однородную среду с проницаемостью $\eps_d$ решение линейной задачи, описывающей колебания на частоте внешнего поля хорошо известно (см. например []), и выражается через сферические функции Бесселя $j_n)$. Как можно показать, выражения для потенциала и плотности заряда в этом случае имеют следующий вид
\begin{equation} 
	C= \frac{-3\eps_d E_0}{\eps + 2\eps_d [1 + (\eps/\eps_\infty - 1) G_1 ]},  
\end{equation}
\begin{equation} 
\rho_{01} = \frac{\eps -1}{4\pi}k_{p1}\frac{C}{j_1' (\kp_1)}, 	
\end{equation}
\begin{equation} 
	\rho_1 = C \frac{-k_{p1}^2a\w_p^2}{4\pi\w(\w - i \nu) } \frac{j_1(k_{p1} r)}{\kp_1 j_1' (\kp_1)}\cos\theta, \quad \ph_1 = C r + \frac{4\pi \rho_1 }{\kp_1^2 \eps_\infty},
\end{equation}

УБРАТЬ ОТДЕЛЬНУЮ КАППА?
где $a$ — радиус сферы, $\theta$ и $r$  — полярный угол и радиус, $G_1 ={j_1(\kp_1)}/{\kp_1 j_1'(\kp_1)}$, $\kp_1= k_p(\omega)a$, $k_{p1,2} = \sqrt{[\w_{1,2}(\w_{1,2} - i\nu) - \w_p^2/\eps_\infty]/v_0^2} $, $\eps = \eps_\infty - {\w_p^2}/{\w(\w - i\nu)}$. Последнее из перечисленных величин имеет смысл диэлектрической проницаемости металла в отсутствие нелокальности. 
Положение наиболее сильного из них, дипольного поверхностного плазмона (резонанс Ми), без учета пространственной дисперсии, зависит от диэлектрической проницаемости внешней среды, определяется выражением $ \eps + 2\eps_d \approx 0$, и частота генерируемой в наночастице второй гармоники колебаний может лежать в области частот отвечающей возможности возбуждения объемных плазмонов. Значения их резонансных частот определяются общим дисперсионным уравнением:
\begin{equation} 	
m \eps + \eps_d(m+1)(1 + m (\eps/\eps_\infty - 1) G_m) = 0,	
\end{equation}
ПОСМОТРЕТЬ ИНДЕКС G. ПРИ M=0,2 ИНДЕКС КАППА НЕ РАВЕН 0,2
($m$ – номер мультиполя), которое может быть также получено из решения однородной краевой задачи () в отсутствие внешнего поля. В интересующем нас случае слабой пространственной дисперсии $r_0 << a$ значения резонансных частот слабо зависят от параметров окружающей среды и приближенно могут быть найдены из соотношения $ \kp_1 \approx \eta_{m+1}^k$, где $\eta_{m+1}^k$ $k$-й корень сферической функции Бесселя порядка $m+1$.	Из всех возможных условий двойных резонансов здесь представляет интерес рассмотрение случая с m=0 и m=2 (монопольные и квадрупольные объемные резонансы соответственно ), поскольку в случае сферической наночастицы, как можно увидеть из соотношений () (), источники поля второй гармоники могут возбуждать только колебания монопольного и квадрупольного типов. 

На рисунке () проиллюстрированы положения частот резонансов от диэлектрической проницаемости, при типичных для металических наночастиц значениях параметров nu, Vf, Wp ==. ПРО ВОЗМОЖНОЕ ЧИСЛО ДВОЙНЫХ РЕЗОНАНСОВ И УСЛОВИЯ ДЛЯ ИХ ВОЗНИКНОВЕНИЯ (ПРО НЮ)

На основании решения краевых задач для трех мультипольных составляющих потенциала и плотности заряда полная средняя за период мощность потерь может быть рассчитана как $Q = Q_{dip} + Q_{mono} + Q_{quad}$, где  содержит вклады от дипольных колебаний на основной частоте ($Q_d$) и монопольных и квадрупольных колебаний на удвоенной частоте внешнего поля ($Q_{mono}$, $m=0$ и $Q_{quad}$, $m=2$ соответственно). Более подробное описание расчета мощности потерь описано в приложении. Помимо этого, для дипольных плазмонов необходимо учесть дополнительные потери, обусловленные поверхностными потерями. Для этого эффективную частоту соударений электронов дипольных плазмонов можно представить в виде $\nu_{dip} = \nu + 3/4 v_f / a / \w_p$ [].

На рисунке () представлены зависимости мощности потерь от частоты при различных значениях проницаемостей epsInf и epsD. Сплошной линией указана полная мощность потерь, dot – вклад в потери от дипольных колебаний, пунктир и пунктир с точкой вклад от монопольных и квадрупольных колебаний соответственно. 

Из приведенных графиков видно, что дополнительные резонансы не проявляются в виде отдельных пиков на фоне основных потерь энергии, однако из-за этого увеличивается суммарная мощность потерь. Стоит отметить влияние монопольных резонансов, которые не проявляются в лазерной спектроскопии, так как потенциал монопольных колебаний не выходит за границы частицы, а также не возбуждаются однородным полем. Так же, в некоторых случаях происходит уширение линии потерь. При этом чем ближе резонансная частота находится к удвоенной частоте первой гармоники, тем больший вклад в потери вносит тот или иной тип колебаний.

Чтобы показать, насколько восприимчивы двойные резонансы к параметрам внешней среды можно построить зависимость максимального значения потерь от диэлектрической проницаемости внешней среды. На рисунке () представлены результаты расчетов для сферической наночастицы натрия ……  Sodium cluster and field params 

В практических задачах чаще сталкиваются с наночастицами покрытыми слоем диэлектрика, а не находящимися в сплошной среде, как представлено в данной работе. Однако, модифицируя уравнения () – (), можно получить следующее дисперсионное уравнение для наночастицы в слое диэлектрика толщиной $b$:
\begin{equation} 	
	\eps + \eps_d\frac{m+1}{m}  \frac{1-K_m}{1 + (m+1)K_m/m} = 0, \quad K_m = (\frac{a}{b})^{2m+1} \frac{\eps_d - 1}{\eps_d + (m+1)/m}
\end{equation}

\section{ЗАКЛЮЧЕНИЕ}

В работе продемонстрировано, что в сферических металлических наноструктурах возможно возбуждение двойных плазмонных резонансов, включающих поверхностные плазмоны на основной частоте и объемные плазмоны на удвоенной частоте. Это явление обусловлено нелинейными эффектами, которые усиливаются благодаря резонансным условиям. Результаты показывают, что такие резонансы приводят к увеличению общей мощности поглощения энергии наночастицей, а также могут влиять на уширение спектральных линий. Интерес так же представляет возбуждение монопольных колебаний, которые обычно слабо проявляются. С практической стороны, благодаря эффекту двойного резонанса и высокой чувствительности к параметрам внешней среды наночастицы могут служить источниками излучения для нужд диагностики оптических сред и спектроскопии.

\section{ПРИЛОЖЕНИЕ}

\section{ЛИТЕРАТУРА}

Ссылки на список литературы в тексте приводятся в квадратных скобках [3, 5--9]. Нумерация должна соответствовать порядку упоминания источников в тексте. Каждый пункт в списке литературы должен содержать только один источник. Не допускается указывать в одном пункте списка несколько статей (в том числе несколько самостоятельных частей одной статьи). Ссылки на конкретную страницу, раздел, формулу в~цитируемом источнике даются следующим образом: [1, с. 5], [1, раздел 2], [1, формула~(4)].

Образцы ссылок на типовые источники (статьи в журналах, книги и т. д.) приведены ниже, в разделе <<Список литературы>>. Общие принципы построения ссылок следующие:

--- для статей в периодических изданиях и сборниках обязательно следует указывать DOI (при наличии);

--- если авторов (редакторов) в цитируемом источнике четверо или менее, то указываются все. Если их пятеро или более, то указываются первые трое и дописывается <<и~др.>> / <<, et al.>>.

\section{РИСУНКИ И ТАБЛИЦЫ}

Статья может содержать рисунки и таблицы. Их следует размещать на отдельных страницах в конце документа.

Если рисунок представляет собой какой-либо график, диаграмму и т. д., то его желательно предоставить в векторном формате (eps, pdf и т. п.), воспользовавшись соответствующими опциями экспорта из используемой математической или лабораторной программной среды. При невозможности предоставить рисунок в векторном формате принимается и растровый вариант (png, bmp и т. п.), с которого желательно убрать координатную сетку.

Если рисунок представляет собой фотографию установки, образца и т. п., то его нужно предоставить в растровом формате.

Ширина рисунков должна быть равна 80 или 160 мм. Разрешение растровых рисунков должно быть не хуже 300 точек на дюйм (300 dpi). К печати принимаются цветные рисунки.

При наборе таблиц величины одной размерности следует располагать в столбцах, а~не в строках (если позволяет размер таблицы).

Физические величины в названиях осей и шкал на графиках, а также в столбцах таблиц следует обозначать буквами (как в формулах): например, вместо <<скорость (м/с)>> следует писать <<$V$, м/с>>.

\section{БЛАГОДАРНОСТИ}

Абзац с благодарностями размещается после основного текста (после заключения или выводов):

Работа выполнена при поддержке Российского научного фонда (проект 24-02-00000), Совета по грантам Президента РФ для государственной поддержки молодых российских учёных и по государственной поддержке ведущих научных школ РФ (проект МК-0000.2024.1.2), Министерства науки и высшего образования РФ в рамках государственного задания ИПФ РАН (проект FFUF-2024-0000).

\section*{\hfill ПРИЛОЖЕНИЕ}

Статья может включать одно или несколько приложений, которые размещаются между абзацем с благодарностями и списком литературы. Если приложение одно, то оно именуется <<ПРИЛОЖЕНИЕ>>. Нумерация формул в нём отдельная: (П1), (П2), (П3). Если приложений два или более, то они именуются <<ПРИЛОЖЕНИЕ~1>>, <<ПРИЛОЖЕНИЕ~2>> и т. д. В каждом таком приложении нумерация формул отдельная: (П1.1), (П1.2), (П1.3) в приложении 1, (П2.1), (П2.2), (П2.3) в приложении 2 и так далее. В~дополнение к слову <<ПРИЛОЖЕНИЕ>> названия приложений могут включать также соответствующие смысловые заголовки (как у обычных разделов).

\begin{thebibliography}{99}

\bibitem{1}
Гинзбург В.\,Л. Распространение электромагнитных волн в плазме. Изд. 2-е, перераб. М.~: Наука, 1967. 684~с.
\textcolor{red}{[ссылка на монографию; издание указывается при необходимости; указывается общее число страниц]}

\bibitem{2}
Плазменная гелиогеофизика. В 2 т. Т. 2 / под ред. Л.\,М.~Зеленого, И.\,С.~Веселовского. М.~: Физматлит, 2008. 560~с.
\textcolor{red}{[ссылка на коллективную монографию; указывается общее число страниц]}

\bibitem{3}
Шкляр Д.\,Р. // Плазменная гелиогеофизика. В~2~т. Т.~2 / под ред. Л.\,М.~Зеленого, И.\,С.~Веселовского. М.~: Физматлит, 2008. С.~390--553.
\textcolor{red}{[ссылка на отдельную статью в коллективной монографии; название статьи опускается; указывается соответствующий диапазон страниц; DOI статьи при наличии указывается обязательно]}

\bibitem{4}
Balanis C.\,A. Advanced engineering electromagnetics. 2nd ed. Hoboken~: Wiley, 2012. 1040~p.
\textcolor{red}{[ссылка на англоязычную монографию; издание указывается при необходимости; указывается общее число страниц]}

\bibitem{5}
The THEMIS mission / ed. by J.\,L.~Burch, V.~Angelopoulos. New York~: Springer-Verlag, 2009. 587~p.
\textcolor{red}{[ссылка на англоязычную коллективную монографию; указывается общее число страниц]}

\bibitem{6}
Harvey P., Taylor E., Sterling R., Cully M. // The THEMIS mission / ed. by J.\,L.~Burch, V.~Angelopoulos. New York~: Springer-Verlag, 2009. P.~117--152. doi: 10.1007/s11214-008-9416-2
\textcolor{red}{[ссылка на отдельную статью в англоязычной коллективной монографии; название статьи опускается; указывается соответствующий диапазон страниц; DOI статьи при наличии указывается обязательно]}

\bibitem{7}
Бабин А.\,А., Киселев А.\,М., Правденко К.\,И. и др. // Успехи физ. наук. 1999. Т.~169, №~1. С.~80--84. doi: 10.3367/UFNr.0169.199901l.0080
\textcolor{red}{[ссылка на статью в журнале; название статьи опускается; название журнала приводится в сокращённом виде; указывается соответствующий диапазон страниц; DOI статьи при наличии указывается обязательно]}

\bibitem{8}
Balmain K.\,G. // IEEE Trans. Antennas Propag. 1964. V.~12, No.~5. P.~605--617. doi: 10.1109/TAP.1964.1138278
\textcolor{red}{[ссылка на статью в англоязычном журнале; название статьи опускается; название журнала приводится в сокращённом виде; указывается соответствующий диапазон страниц; DOI статьи при наличии указывается обязательно]}

\bibitem{9}
Myers D.\,J., Espenlaub A.\,C., Gelzinyte K., et al. // Appl. Phys. Lett. 2020. V.~116, No.~9. Art. no. 091102. doi: 10.1063/1.5125605
\textcolor{red}{[ссылка на статью в англоязычном журнале, где вместо нумерации страниц используется нумерация статей; название статьи опускается; название журнала приводится в сокращённом виде; DOI статьи при наличии указывается обязательно]}

\bibitem{10}
Шарыкин И.\,Н., Зимовец И.\,В. // 14-я ежегодная конференция <<Физика плазмы в~Солнечной системе>>. 11--15 февраля 2019~г., Москва, Россия. С.~72.
\textcolor{red}{[ссылка на статью в сборнике материалов конференции; название статьи опускается; указываются дата и место проведения; указывается соответствующий диапазон страниц; DOI статьи при наличии указывается обязательно]}

\bibitem{11}
Macotela E.\,L., Clilverd M., Manninen J. // VERSIM 2018 Workshop. Abstracts. 19--23 March 2018, Apatity, Russia. P.~3.
\textcolor{red}{[ссылка на статью в англоязычном сборнике материалов конференции; название статьи опускается; указываются дата и место проведения; указывается соответствующий диапазон страниц; DOI статьи при наличии указывается обязательно]}

\bibitem{12}
Баханов В.\,В., Демакова А.\,А., Зуйкова Э.\,М. Определение спектров короткомасштабных ветровых волн оптическим методом~: препринт №~814. Нижний Новгород~: Ин-т прикладной физики РАН, 2017. 8~с.
\textcolor{red}{[ссылка на препринт; указывается общее число страниц]}

\bibitem{13}
Poggio A.\,J., Adams R.\,W. Approximations for terms related to the kernel in thin-wire integral equations~: techn. rep. AFWL-TR-76-98. Livermore~: Lawrence Livermore Laboratory, 1977. 44~p.
\textcolor{red}{[ссылка на англоязычный технический отчёт; указывается общее число страниц]}

\bibitem{14}
Beasley M.\,A. https://arxiv.org/abs/2003.04093
\textcolor{red}{[ссылка на препринт в arXiv]}

\bibitem{15}
Зотова И.\,В. Генерация, усиление и нелинейная трансформация импульсов сверхизлучения релятивистскими электронными пучками и сгустками~: дис.~... д-ра физ.-мат. наук. Нижний Новгород, 2014. 291~с.
\textcolor{red}{[ссылка на диссертацию; указывается общее число страниц]}

\bibitem{16}
Манаков С.\,А. Экспериментальные исследования структурно-неоднородных сред методами когерентной акустики~: автореф. дис.~... канд. физ.-мат. наук. Нижний Новгород, 2016. 24~с.
\textcolor{red}{[ссылка на автореферат диссертации; указывается общее число страниц]}

\bibitem{17}
Manninen J. Some aspects of ELF-VLF emissions in geophysical research~: PhD thesis. Oulu, 2005. 194~p.
\textcolor{red}{[ссылка на англоязычную диссертацию; указывается общее число страниц]}

\bibitem{18}
https://radiophysics.unn.ru/
\textcolor{red}{[ссылка на интернет-ресурс]}

\bibitem{19}
Пат. 2637215 РФ, МПК B02C 19/16 (2006.01), B02C 17/00 (2006.01). Вибрационная мельница~: №~2017105030~: заявл. 15.02.2017~: опубл. 01.12.2017~/ Артеменко~К.\,И., Богданов~Н.\,Э.~; заявитель БГТУ. 8~с.
\textcolor{red}{[ссылка на патент РФ; индексы МПК/МКИ указываются при наличии; заявитель указывается при необходимости; указывается общее число страниц]}

\bibitem{20}
Пат. 6147647 США, МПК H01Q 1/38. Circularly polarized dielectric resonator antenna~: №~09/150157~: заявл. 09.09.1998~: опубл. 14.11.2000~/ Tassoudji~M.\,A., Ozaki~E.\,T., Lin Y.\,C. 14~с.
\textcolor{red}{[ссылка на патент США; индексы МПК/МКИ указываются при наличии; заявитель указывается при необходимости; указывается общее число страниц]}

\bibitem{21}
Авт. свид. 1007970 СССР, МКИ В25J 15/00. Устройство для захвата неориентированных деталей типа валов~: №~3360585/25-08~: заявл. 23.11.1981~: опубл. 30.03.1983~/ Ваулин~В.\,С., Кемайкин~В.\,Г. 2~с.
\textcolor{red}{[ссылка на авторское свидетельство; индексы МПК/МКИ указываются при наличии; указывается общее число страниц]}

\bibitem{22}
ГОСТ Р 51771-2001. Аппаратура радиоэлектронная бытовая. Входные и выходные параметры и типы соединений. Технические требования. М.~: Госстандарт России, 2001. 31~с.
\textcolor{red}{[ссылка на стандарт; указывается общее число страниц]}

\bibitem{23}
ISO 26324:2012. Information and documentation. Digital object identifier system. Geneva~: ISO, 2012. 24~p.
\textcolor{red}{[ссылка на англоязычный стандарт; указывается общее число страниц]}

\end{thebibliography}

\end{document}
