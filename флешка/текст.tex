\documentclass[12pt, a4paper]{article}

\usepackage[T1, T2A]{fontenc}
\usepackage[utf8]{inputenc}
\usepackage[english, russian]{babel}
\usepackage{amsmath, amssymb}
\usepackage[left=3cm, right=1.5cm, top=2cm, bottom=2cm]{geometry}
\usepackage[displaymath]{lineno}
\usepackage[space]{cite}
\usepackage{xcolor, graphicx}

\renewcommand{\vec}{\mathbf}
\linenumbers
\linespread{2}

\begin{document}
\thispagestyle{empty}

Контактные данные автора, ответственного за связь с редакцией\\
Иванов Иван Иванович\\
Институт прикладной физики им. А. В. Гапонова-Грехова РАН, 603950, г. Нижний Новгород, БОКС-120, ул. Ульянова, 46\\
контактный телефон +7 000 111-11-11\\
e-mail: author@example.ru

\newpage
\setcounter{page}{1}

УДК 111.111 --- вставьте подходящий индекс УДК, см. https://teacode.com/online/udc/

{\large\bf НАЗВАНИЕ СТАТЬИ НА РУССКОМ ЯЗЫКЕ}

И.\,И. Иванов$^1$, С.\,С. Сафина$^{1,2}$, П.\,П. Петров$^3$

$^1$ Институт прикладной физики им. А.\,В. Гапонова-Грехова РАН, г. Нижний Новгород;\\
$^2$ Нижегородский госуниверситет им. Н.\,И. Лобачевского, г. Нижний Новгород;\\
$^3$ Национальный исследовательский ядерный университет <<МИФИ>>, г. Москва, Россия\\

Здесь печатается аннотация статьи на русском языке.

\newpage

{\large\bf TITLE IN ENGLISH}

I.\,I. Ivanov, S.\,S. Safina, and P.\,P. Petrov

The abstract in English should be translated from the abstract in Russian.\\

Здесь желательно привести перевод часто используемых в статье специальных терминов на английский язык:\\
захваченные частицы --- trapped particles

\newpage

\section*{В\,В\,Е\,Д\,Е\,Н\,И\,Е}

Рукопись желательно набирать в формате \LaTeX. Принимаются также рукописи в форматах docx и odt.

Максимальный объём текста зависит от типа публикации: мини-обзор~--- не более 50 страниц, регулярная статья~--- не более 28 страниц, письмо~--- не более 12 страниц, комментарий~--- не более 6 страниц. Рекомендуемое число рисунков также определяется типом публикации: мини-обзор~--- не более 12 рисунков, регулярная статья~--- не более 8 рисунков, письмо~--- не более 3 рисунков, комментарий~--- не более 1 рисунка.

\section{ОСНОВНОЙ ТЕКСТ}

Аббревиатуры необходимо расшифровывать при первом использовании, даже самые распространённые (КПД, СВЧ и т. д.). Не следует использовать много разных аббревиатур.

Математические выражения и формулы в основном тексте работы нумеруются сквозным образом: (1), (2), (3) и т. д. Скалярные величины набираются курсивом, векторы и матрицы~--- прямым жирным шрифтом без стрелок: $\vec{F} = q\vec{E}$. Не рекомендуется использовать кириллицу в индексах: например, вместо $d_\textup{ЛОВ}$ и $t_\textup{нагр}$ желательно писать $d_\mathrm{BWO}$ и $t_\mathrm{heat}$ соответственно. Все используемые обозначения величин необходимо расшифровывать при первом использовании, даже общепринятые ($c$~--- скорость света).

\section{ЛИТЕРАТУРА}

Ссылки на список литературы в тексте приводятся в квадратных скобках [3, 5--9]. Нумерация должна соответствовать порядку упоминания источников в тексте. Каждый пункт в списке литературы должен содержать только один источник. Не допускается указывать в одном пункте списка несколько статей (в том числе несколько самостоятельных частей одной статьи). Ссылки на конкретную страницу, раздел, формулу в~цитируемом источнике даются следующим образом: [1, с. 5], [1, раздел 2], [1, формула~(4)].

Образцы ссылок на типовые источники (статьи в журналах, книги и т. д.) приведены ниже, в разделе <<Список литературы>>. Общие принципы построения ссылок следующие:

--- для статей в периодических изданиях и сборниках обязательно следует указывать DOI (при наличии);

--- если авторов (редакторов) в цитируемом источнике четверо или менее, то указываются все. Если их пятеро или более, то указываются первые трое и дописывается <<и~др.>> / <<, et al.>>.

\section{РИСУНКИ И ТАБЛИЦЫ}

Статья может содержать рисунки и таблицы. Их следует размещать на отдельных страницах в конце документа.

Если рисунок представляет собой какой-либо график, диаграмму и т. д., то его желательно предоставить в векторном формате (eps, pdf и т. п.), воспользовавшись соответствующими опциями экспорта из используемой математической или лабораторной программной среды. При невозможности предоставить рисунок в векторном формате принимается и растровый вариант (png, bmp и т. п.), с которого желательно убрать координатную сетку.

Если рисунок представляет собой фотографию установки, образца и т. п., то его нужно предоставить в растровом формате.

Ширина рисунков должна быть равна 80 или 160 мм. Разрешение растровых рисунков должно быть не хуже 300 точек на дюйм (300 dpi). К печати принимаются цветные рисунки.

При наборе таблиц величины одной размерности следует располагать в столбцах, а~не в строках (если позволяет размер таблицы).

Физические величины в названиях осей и шкал на графиках, а также в столбцах таблиц следует обозначать буквами (как в формулах): например, вместо <<скорость (м/с)>> следует писать <<$V$, м/с>>.

\section{БЛАГОДАРНОСТИ}

Абзац с благодарностями размещается после основного текста (после заключения или выводов):

Работа выполнена при поддержке Российского научного фонда (проект 24-02-00000), Совета по грантам Президента РФ для государственной поддержки молодых российских учёных и по государственной поддержке ведущих научных школ РФ (проект МК-0000.2024.1.2), Министерства науки и высшего образования РФ в рамках государственного задания ИПФ РАН (проект FFUF-2024-0000).

\section*{\hfill ПРИЛОЖЕНИЕ}

Статья может включать одно или несколько приложений, которые размещаются между абзацем с благодарностями и списком литературы. Если приложение одно, то оно именуется <<ПРИЛОЖЕНИЕ>>. Нумерация формул в нём отдельная: (П1), (П2), (П3). Если приложений два или более, то они именуются <<ПРИЛОЖЕНИЕ~1>>, <<ПРИЛОЖЕНИЕ~2>> и т. д. В каждом таком приложении нумерация формул отдельная: (П1.1), (П1.2), (П1.3) в приложении 1, (П2.1), (П2.2), (П2.3) в приложении 2 и так далее. В~дополнение к слову <<ПРИЛОЖЕНИЕ>> названия приложений могут включать также соответствующие смысловые заголовки (как у обычных разделов).

\begin{thebibliography}{99}

\bibitem{1}
Гинзбург В.\,Л. Распространение электромагнитных волн в плазме. Изд. 2-е, перераб. М.~: Наука, 1967. 684~с.
\textcolor{red}{[ссылка на монографию; издание указывается при необходимости; указывается общее число страниц]}

\bibitem{2}
Плазменная гелиогеофизика. В 2 т. Т. 2 / под ред. Л.\,М.~Зеленого, И.\,С.~Веселовского. М.~: Физматлит, 2008. 560~с.
\textcolor{red}{[ссылка на коллективную монографию; указывается общее число страниц]}

\bibitem{3}
Шкляр Д.\,Р. // Плазменная гелиогеофизика. В~2~т. Т.~2 / под ред. Л.\,М.~Зеленого, И.\,С.~Веселовского. М.~: Физматлит, 2008. С.~390--553.
\textcolor{red}{[ссылка на отдельную статью в коллективной монографии; название статьи опускается; указывается соответствующий диапазон страниц; DOI статьи при наличии указывается обязательно]}

\bibitem{4}
Balanis C.\,A. Advanced engineering electromagnetics. 2nd ed. Hoboken~: Wiley, 2012. 1040~p.
\textcolor{red}{[ссылка на англоязычную монографию; издание указывается при необходимости; указывается общее число страниц]}

\bibitem{5}
The THEMIS mission / ed. by J.\,L.~Burch, V.~Angelopoulos. New York~: Springer-Verlag, 2009. 587~p.
\textcolor{red}{[ссылка на англоязычную коллективную монографию; указывается общее число страниц]}

\bibitem{6}
Harvey P., Taylor E., Sterling R., Cully M. // The THEMIS mission / ed. by J.\,L.~Burch, V.~Angelopoulos. New York~: Springer-Verlag, 2009. P.~117--152. doi: 10.1007/s11214-008-9416-2
\textcolor{red}{[ссылка на отдельную статью в англоязычной коллективной монографии; название статьи опускается; указывается соответствующий диапазон страниц; DOI статьи при наличии указывается обязательно]}

\bibitem{7}
Бабин А.\,А., Киселев А.\,М., Правденко К.\,И. и др. // Успехи физ. наук. 1999. Т.~169, №~1. С.~80--84. doi: 10.3367/UFNr.0169.199901l.0080
\textcolor{red}{[ссылка на статью в журнале; название статьи опускается; название журнала приводится в сокращённом виде; указывается соответствующий диапазон страниц; DOI статьи при наличии указывается обязательно]}

\bibitem{8}
Balmain K.\,G. // IEEE Trans. Antennas Propag. 1964. V.~12, No.~5. P.~605--617. doi: 10.1109/TAP.1964.1138278
\textcolor{red}{[ссылка на статью в англоязычном журнале; название статьи опускается; название журнала приводится в сокращённом виде; указывается соответствующий диапазон страниц; DOI статьи при наличии указывается обязательно]}

\bibitem{9}
Myers D.\,J., Espenlaub A.\,C., Gelzinyte K., et al. // Appl. Phys. Lett. 2020. V.~116, No.~9. Art. no. 091102. doi: 10.1063/1.5125605
\textcolor{red}{[ссылка на статью в англоязычном журнале, где вместо нумерации страниц используется нумерация статей; название статьи опускается; название журнала приводится в сокращённом виде; DOI статьи при наличии указывается обязательно]}

\bibitem{10}
Шарыкин И.\,Н., Зимовец И.\,В. // 14-я ежегодная конференция <<Физика плазмы в~Солнечной системе>>. 11--15 февраля 2019~г., Москва, Россия. С.~72.
\textcolor{red}{[ссылка на статью в сборнике материалов конференции; название статьи опускается; указываются дата и место проведения; указывается соответствующий диапазон страниц; DOI статьи при наличии указывается обязательно]}

\bibitem{11}
Macotela E.\,L., Clilverd M., Manninen J. // VERSIM 2018 Workshop. Abstracts. 19--23 March 2018, Apatity, Russia. P.~3.
\textcolor{red}{[ссылка на статью в англоязычном сборнике материалов конференции; название статьи опускается; указываются дата и место проведения; указывается соответствующий диапазон страниц; DOI статьи при наличии указывается обязательно]}

\bibitem{12}
Баханов В.\,В., Демакова А.\,А., Зуйкова Э.\,М. Определение спектров короткомасштабных ветровых волн оптическим методом~: препринт №~814. Нижний Новгород~: Ин-т прикладной физики РАН, 2017. 8~с.
\textcolor{red}{[ссылка на препринт; указывается общее число страниц]}

\bibitem{13}
Poggio A.\,J., Adams R.\,W. Approximations for terms related to the kernel in thin-wire integral equations~: techn. rep. AFWL-TR-76-98. Livermore~: Lawrence Livermore Laboratory, 1977. 44~p.
\textcolor{red}{[ссылка на англоязычный технический отчёт; указывается общее число страниц]}

\bibitem{14}
Beasley M.\,A. https://arxiv.org/abs/2003.04093
\textcolor{red}{[ссылка на препринт в arXiv]}

\bibitem{15}
Зотова И.\,В. Генерация, усиление и нелинейная трансформация импульсов сверхизлучения релятивистскими электронными пучками и сгустками~: дис.~... д-ра физ.-мат. наук. Нижний Новгород, 2014. 291~с.
\textcolor{red}{[ссылка на диссертацию; указывается общее число страниц]}

\bibitem{16}
Манаков С.\,А. Экспериментальные исследования структурно-неоднородных сред методами когерентной акустики~: автореф. дис.~... канд. физ.-мат. наук. Нижний Новгород, 2016. 24~с.
\textcolor{red}{[ссылка на автореферат диссертации; указывается общее число страниц]}

\bibitem{17}
Manninen J. Some aspects of ELF-VLF emissions in geophysical research~: PhD thesis. Oulu, 2005. 194~p.
\textcolor{red}{[ссылка на англоязычную диссертацию; указывается общее число страниц]}

\bibitem{18}
https://radiophysics.unn.ru/
\textcolor{red}{[ссылка на интернет-ресурс]}

\bibitem{19}
Пат. 2637215 РФ, МПК B02C 19/16 (2006.01), B02C 17/00 (2006.01). Вибрационная мельница~: №~2017105030~: заявл. 15.02.2017~: опубл. 01.12.2017~/ Артеменко~К.\,И., Богданов~Н.\,Э.~; заявитель БГТУ. 8~с.
\textcolor{red}{[ссылка на патент РФ; индексы МПК/МКИ указываются при наличии; заявитель указывается при необходимости; указывается общее число страниц]}

\bibitem{20}
Пат. 6147647 США, МПК H01Q 1/38. Circularly polarized dielectric resonator antenna~: №~09/150157~: заявл. 09.09.1998~: опубл. 14.11.2000~/ Tassoudji~M.\,A., Ozaki~E.\,T., Lin Y.\,C. 14~с.
\textcolor{red}{[ссылка на патент США; индексы МПК/МКИ указываются при наличии; заявитель указывается при необходимости; указывается общее число страниц]}

\bibitem{21}
Авт. свид. 1007970 СССР, МКИ В25J 15/00. Устройство для захвата неориентированных деталей типа валов~: №~3360585/25-08~: заявл. 23.11.1981~: опубл. 30.03.1983~/ Ваулин~В.\,С., Кемайкин~В.\,Г. 2~с.
\textcolor{red}{[ссылка на авторское свидетельство; индексы МПК/МКИ указываются при наличии; указывается общее число страниц]}

\bibitem{22}
ГОСТ Р 51771-2001. Аппаратура радиоэлектронная бытовая. Входные и выходные параметры и типы соединений. Технические требования. М.~: Госстандарт России, 2001. 31~с.
\textcolor{red}{[ссылка на стандарт; указывается общее число страниц]}

\bibitem{23}
ISO 26324:2012. Information and documentation. Digital object identifier system. Geneva~: ISO, 2012. 24~p.
\textcolor{red}{[ссылка на англоязычный стандарт; указывается общее число страниц]}

\end{thebibliography}

\end{document}
